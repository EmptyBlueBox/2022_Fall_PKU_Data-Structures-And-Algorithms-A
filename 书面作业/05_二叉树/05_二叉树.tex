\documentclass[11pt,a4paper,fleqn]{article}
\usepackage[UTF8]{ctex}
\usepackage[a4paper]{geometry}
\geometry{left=2.0cm,right=2.0cm,top=2.5cm,bottom=2.5cm}

\usepackage{amsmath,amsfonts,graphicx,amssymb,bm,amsthm}
\usepackage{mathrsfs}
\usepackage{algorithm,algorithmicx}
\usepackage{fancyhdr}
\usepackage{tikz}
\usetikzlibrary{intersections,through}
\usetikzlibrary{calc}
\usepackage{caption}
\usepackage{verbatim}
\usepackage{comment}

\renewcommand\thesection{Question \arabic{section}.}

\setlength{\headheight}{14pt}
\setlength{\parindent}{0 in}

\newtheorem{theorem}{Theorem}
\newtheorem{lemma}[theorem]{Lemma}
\newtheorem{proposition}[theorem]{Proposition}
\newtheorem{claim}[theorem]{Claim}
\newtheorem{corollary}[theorem]{Corollary}
\newtheorem{definition}[theorem]{Definition}

\newcommand\E{\mathbb{E}}
\newcommand{\hwid}{5}
\newcommand{\name}{梁昱桐}
\newcommand{\institute}{信息科学技术学院}
\newcommand{\id}{2100013116}
%\newcommand{\order}{324}

\usetikzlibrary{positioning}

\begin{document}

\pagestyle{fancy}
\lhead{Peking University}
\chead{}
\rhead{Principle of Economics, 2022 Fall}

\setlength{\parindent}{0pt}

\begin{center}
	{\LARGE \bf Data Structures and Algorithms A-Binary Tree Home Work \hwid}\\
	{\Large \name}\\
	{\Large \institute}\\
	{\Large \id}\\
	%{\Large \order}\\
\end{center}

\section{}
证明:判断以下叙述是否成立,并给出证明,若不成立,给出反例:
	
已知先序遍历序列和后序遍历序列可以确定唯一一棵二叉树。

%\begin{comment}
\begin{itemize}

	\item[1)]请用供需模型画出征税前市场的均衡。

	\textbf{Solution:}

	\item[2)]请在图上画出征税后新的均衡。征税后市场上新的均衡价格是多少?

    \textbf{Solution:}

	观众的均衡价格是500元,举办方能获得495元

	\item[3)]是举办方还是观众承担了更多的税收?请在图上画出观众和举办方各自的税收负担 (tax burden) ,并解释你的结论。

	\textbf{Solution:}
	
	\item[4)]假定门票价格并不相同:有 10,000 张 VIP 早鸟票,每张门票价格 1,000 元; 另有 25,000 张普通票,每张门票价格 400 元。为了分析的简化,我们假定两个子市场是完全割裂的:购买早鸟票的观众不会因为价格的变动而考虑普通票;同理,购买普通票的观众不会因为价格的变动而考虑早鸟票。假定此时政府仍然决定对每张门票征收 5 元的税收,请计算此时观众和举办方各自的税收负担并与 3)中的结论做比较。

	\textbf{Solution:}
	
\end{itemize}
\section{}


\clearpage
\end{document}